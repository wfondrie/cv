\documentclass[11pt]{article}

% Use Helvetica font.
\usepackage{helvet}
\renewcommand{\familydefault}{\sfdefault}

% Use 1/2-inch margins.
\usepackage[left=0.5in,top=2.0cm,bottom=1.6cm,right=0.5in]{geometry}

% Allow URLs and hyperlinks.
\usepackage{url}
\usepackage[colorlinks=true]{hyperref}

% No page numbers.
\pagestyle{empty}

% Turn off indenting.
\setlength{\parindent}{0in}

% Define the section command.
\newcommand{\mysection}[1]{\vspace{1ex}{\bf #1} \vspace*{0.3ex}}

% Exclude stuff that doesn't belong in the biosketch.
\newcommand{\notbiosketch}[1]{}

\newcommand{\ul}[1]{\underline{#1}}

\begin{document}
\setcounter{page}{1}
\begin{tabular}{p{\columnwidth}}
  \hline
  \centerline {\bf BIOGRAPHICAL SKETCH} \\
  \hline
  {\sc Name}: William Ellis Fondrie \\
  \hline
  {\sc Position Title}: Senior Fellow \\
  \hline
  e{\sc RA Commons user name}: WFONDRIE  \\
  \hline
  {\sc EDUCATION/TRAINING} \\
\end{tabular}

\begin{tabular}{p{2.2in}|p{0.8in}|p{1in}|p{1.1in}|p{1.75in}}
  \hline
  \centerline{{\small {\sc Institution and location}}} &
  \centerline{{\small {\sc Degree}}} &
  \centerline{{\small {\sc Start Date}}} &
  \centerline{{\small {\sc Completion Date}}} &                                           
  \centerline{{\small {\sc Field of Study}}} \\
  \hline
  University of North Carolina at Chapel Hill & B.S.\ & 08/2009 & 05/2013 &  Chemistry \\
  University of Maryland, Baltimore & Ph.D.\ & 08/2013 & 06/2018 & Molecular Medicine\\
  University of Washington & Postdoc & 07/2018 & present & Computational Biology \\
\end{tabular}

\mysection{A. Personal Statement}
% Briefly describe why you are well-suited for your role(s) in this project.
% Relevant factors may include: aspects of your training; your previous
% experimental work on this specific topic or related topics; your technical
% expertise; your collaborators or scientific environment; and/or your past
% performance in this or related fields.

% You may cite up to four publications or research products that highlight
% your experience and qualifications for this project. Research products can
% include, but are not limited to, audio or video products; conference proceedings
% such as meeting abstracts, posters, or other presentations; patents; data and
% research materials; databases; educational aids or curricula; instruments or
% equipment; models; protocols; and software or netware.

My proposal seeks to advance computational methods for characterizing proteins in Alzheimer's disease by tandem mass spectrometry. I was first introduced to mass spectrometry as a graduate student at UMB, and the ability to measure thousands of analytes in a single experiment immediately captured my interest---it was the ideal vehicle to engage my undergraduate chemistry training in biomedical research. My dissertation research provided foundational training in both proteomics and in the mass spectrometry analysis of lipids across biological domains that include cancer, infectious disease, and cardiovascular disease. These efforts led to contributions in nine peer-reviewed publications during my graduate career.\\

\indent During this time, I found that my research interests lie in developing methods to translate the acquired mass spectra into biological insight. This interest in the computational analysis of mass spectra led me to learn many computational biology skills in an {\it ad hoc} fashion, including the basics of machine learning and statistical modeling. These interests  led me to Dr.\ William Noble's lab in the department of Genome Sciences at UW for my postdoctoral training. Dr.\ Noble is one of the foremost researchers in computational methods for mass spectrometry and genomics. The department also houses Dr.\ Michael MacCoss, a pioneer in the mass spectrometry field and my co-sponsor for this proposal.\\

\indent My goal is for the training provided by this fellowship to prepare me for a career as an independent researcher with the aim to improve human health through the computational analysis of mass spectra. The proposed project will provide formal training in new facets of computational biology, including deep learning, algorithms, and software design, and a cutting-edge mass spectrometry technique for proteomics. Furthermore, these studies will be my first exposure to the study of Alzheimer's disease. With the additional professional skills that will be honed over the course of the project, this fellowship will enable me to acheive my goal.

\mysection{B. Positions and Honors}
% List in chronological order the positions you've held that are relevant to this
% application, concluding with your present position. High school students and
% undergraduates may include any previous positions. For individuals who are not
% currently located at the applicant organization, include the expected position
% at the applicant organization and the expected start date.

% List any relevant academic and professional achievements and honors. In particular,
% Students, postdoctorates, and junior faculty should include scholarships,
% traineeships, fellowships, and development awards, as applicable.

{\bf Positions} \\
\begin{tabular}{p{0.75in}p{6.5in}}
  2012--13 & Undergraduate Research Assistant, UNC Chapel Hill\\
           & Adviser: John Papanikolas, Ph.D. \\
  2013--18 & Graduate Research Assistant, UM Baltimore\\
           & Advisers: David Goodlett, Ph.D.\ and Dudley Strickland, Ph.D.\\
  2018--   & Senior Fellow, UW Genome Sciences\\
           & Adviser: William Noble, Ph.D.\\
\end{tabular}

\vspace{1ex}{\bf Academic and Professional Honors} \\
\begin{tabular}{p{0.75in}p{6.5in}}
  2009 & Central Carolina's chapter of Phi Beta Kappa Scholarship \\
  2012 & Markham Summer Undergraduate Research Award \\
  2016 & Ruth L. Kirschstein Institutional NRSA. NIH 2T32HL007698. \em{Predoctoral Trainee} \\
  2017 & Ruth L. Kirschstein Individual NRSA. NIH 1F31CA213815. \em{Impact Score: 25 (16.0\%)} \\
  2017 & Travel Fellowship to the May Institute on Computation and Statistics for Mass Spectrometry and Proteomics \\
\end{tabular}

\vspace{1ex}{\bf Professional Memberships} \\
\begin{tabular}{p{0.75in}p{6.5in}}
  2014-- & Member, American Society for Mass Spectrometry \\
\end{tabular}

\mysection{C. Contributions to Science}
% For each contribution, indicate the following:
% - the historical background that frames the scientific problem
% - the central finding(s)
% - the influence of the finding(s) on the progress of science or the application
%   of those finding(s) to health or technology
% - and your specific role in the described work.

% For each contribution, you may cite up to four publications or research
% products that are relevant to the contribution. If you are not the author of
% the product, indicate what your role or contribution was. Note that while
% you may mention manuscripts that have not yet been accepted for publication
% as part of your contribution, you may cite only published papers to support
% each contribution. Research products can include audio or video products (see
% the NIH Grants Policy Statement, Section 2.3.7.7: Post-Submission Grant
% Application Materials); conference proceedings such as meeting abstracts,
% posters, or other presentations; patents; data and research materials;
% databases; educational aids or curricula; instruments or equipment; models;
% protocols; and software or netware.

% You may provide a URL to a full list of your published work. This URL must be
% to a Federal Government website (a .gov suffix). NIH recommends using My
% Bibliography. Providing a URL to a list of published work is not required.

{\bf Evaluating Surface Chemistries of Dye-Sensitized Solar Cells (Early Career)} \\
As an undergraduate, I had the privilege to work in the lab of Dr.\ John Papanikolas on projects investigating the photophysics of chromophore assemblies for use as light harvesting antennas in dye-sensitized solar cells. My work involved using steady-state and transient emission spectroscopy to observe the exchange of energy between Ru(II), Os(II) and Coumarin 343 chromophores upon photoexcitation. Additionally, I performed molecular dynamics simulations to investigate how tether length and flexibility in chromophore assemblies impact the energy dynamics. These results have aided in evaluating new light harvesting antenna architectures and the publication of our work in the \textit{Journal of the American Chemical Society}.

\begin{enumerate}
  \item Ma D, Bettis SE, Hanson K, Minakova M, Alibabaei L, {\bf Fondrie W}, Ryan DM, Papoian GA, Meyer TJ, Waters ML, Papanikolas JM. (2013) Interfacial energy conversion in Ru(II) polypyridyl-derivatized oligoproline assemblies on TiO2. \textit{J Am Chem Soc} 135(14):5250--5253. PMID: 23514453.
  \item {\bf Fondrie WE}, Bettis S, Ma D, Minakova M, Wilger D, Papoian G, Waters M, Papanikolas J. Flexibility matters: The role of scaffold tethers in Ru(II) and Os(II) chromophore separation. {\it Southeastern Regional Meeting of the American Chemical Society, November 14-17, 2012}. Raleigh, NC. Poster
\end{enumerate}

{\bf Assessing Exosomes as a Source of Cancer Biomarkers (Graduate)} \\
I studied the role of exosomes in cancer biology through the use of quantitative shotgun proteomics in the lab of Dr.\ Austin Yang, prior to him joining the NIA as a program director in 2015. Exosomes are 30--100 nm secreted microvesicles of endosomal origin, which have been shown to be indicative of cancer progression and a promising biomarker tool. A major challenge to the use of exosomes for biomarker discovery is the discernment of exosome protein cargo from contaminant proteins, such as those from co-isolating microvesicles. Using quantitative shotgun proteomics and machine learning strategies, we have shown that it is possible to predict the localization of detected proteins to the exosome and that the exosome is a useful tool for biomarker discovery in non-small cell lung cancer.

\begin{enumerate}
  \item Clark DJ*, {\bf Fondrie WE*}, Liao Z, Hanson PI, Fulton A, Mao L, Yang AJ. (2015) Redefining the breast cancer exosome proteome by tandem mass tag quantitative proteomics and multivariate cluster analysis. \textit{Anal Chem} 87(20):10462--10469. PMID: 26378940. {\it (* indicates equal contribution to this work)}
  \item Clark DJ, {\bf Fondrie WE}, Liao Z, Yang AJ, Mao L. (2016) Triple SILAC quantitative proteomic analysis reveals differential abundance of cell signaling proteins between normal and lung cancer-derived exosomes. {\it J Proteomics} 133:161--169. PMID: 26739763.
\end{enumerate}

{\bf Identifying Pathogens from Lipids by Mass Spectrometry (Graduate)} \\
Developing methods to identify microbial pathogens was a major piece of my dissertation research, which was conducted in the labs of Dr.\ David Goodlett and Dr.\ Dudley Strickland. The rapid identification of pathogens in an infection is critical for the efficacious treatment of the patient, particularly if the assay can provide insight into antibiotic resistances. In Gram-negative bacteria, lipopolysaccharide (LPS) is known to be one of the major glycolipid constituents of the outer membrane. Previous studies have shown that there is immense diversity in LPS accross bacterial species, including the membrane anchor component, lipid A. Additionally, the structure of lipid A has been shown to directly impact the susceptability of an organism to colistin, a last-line antibiotic. Using a library we built containing MALDI-TOF mass spectra of membrane glycolipids from 50 microbial species, my work focused on using machine learning to identify species and detect colistin resistance. We found that the classifiers were able to reliably identify bacterial species from the background library and discriminate between colistin-resistant and -susceptible strains, for both isolate and polymicrobial specimens.

\begin{enumerate}
  \item {\bf Fondrie WE}, Liang T, Oyler BL, Leung LM, Ernst RK, Strickland DK, Goodlett DR. (2018) Pathogen identification direct from polymicrobial specimens using membrane glycolipids. {\it Sci Rep} 8(1):15857. PMID: 30367087.
  \item Leung LM, {\bf Fondrie WE}, Doi Y, Johnson JK, Strickland DK, Ernst RK, Goodlett DR. (2017) Identification of the ESKAPE pathogens by mass spectrometric analysis of microbial membrane glycolipids. {\it Sci Rep} 7(1):6403. PMID: 28743946
  \item Liang T, Leung LM, Opene B, {\bf Fondrie WE}, Lee YI, Chandler CE, Yoon SH, Doi Y, Ernst RK, Goodlett DR. (2019) Rapid microbial identification and antibiotic resistance detection by mass spectrometric analysis of membrane lipids. {\it Anal Chem} 91(2):1286--1294. PMID: 30571097
  \item {\bf Fondrie WE}, Leung LM, Strickland DK, Ernst RK, Goodlett DR. Detecting antibiotic resistance by MALDI-TOF analysis of bacterial membrane glycolipids. {\it ASMS Annual Conference on Mass Spectrometry and Allied Topics, June 4, 2017}. Indianapolis, IN. Oral
\end{enumerate}

{\bf Improving the Detection of Cross-Linked Peptides in Proteomics (Postdoctoral)} \\
My primary project since starting as a postdoctoral fellow in Dr.\ William Noble's lab has been on improving computational methods for the analysis of cross-linking mass spectrometry data (XL-MS). XL-MS is a powerful tool to gain insight into protein structures and interactions. In traditional shotgun proteomics, database search post-processing tools have proven useful to improve the sensitivity of peptide detection and calibrate database search score functions. However, these tools are ill-suited to handle the challenges of typical XL-MS experiments, which often yield a limited number of true cross-linked peptide-spectrum matches (PSMs) and fail to account for the specific nuances of confidence estimation. To address these challenges, my work has focused around training neural network models using a diversity of publicly available XL-MS datasets for use as a post-processing tool. Ultimately, I aim to produce an open source tool that increases the reliability of XL-MS peptide detection using models that are pretrained on a wide diversity of data.

\begin{enumerate}
  \item {\bf Fondrie WE}, Noble WS. Robust Cross-Linked Peptide Detection Using Pretrained Neural Networks. {\it ASMS Annual Conference on Mass Spectrometry and Allied Topics, June 2-6, 2019 }. Atlanta, GA. Poster 
  \item {\bf Fondrie WE}, Zelter A, Henry E, Abel S, Le Roch KG, Davis TN, Noble WS. Building Robust Score Vectors for Cross-Linked Peptide Identification. {\it University of Washington Department of Genome Sciences Annual Retreat, Sept 17-19, 2018}. Leavenworth, WA. Poster
\end{enumerate}

A list of my published work is available at: \\
\url{https://www.ncbi.nlm.nih.gov/labs/bibliography/william.fondrie.2/bibliography/public/}

\mysection{D. Research Support and Scholastic Performance}
% List by institution and year all graduate scientific and/or professional
% courses with grades. In addition, explain any grading system used if
% it differs from a 1-100 scale; an A, B, C, D, F system; or a 0-4.0 scale.
% Also indicate the levels required for a passing grade.

{\bf The University of Maryland, Baltimore (Ph.D.)} \\
Courses were graded on an A through F scale, with a B average grade required to pass. \\
{\bf GPA:} 3.99 (out of 4.00) \\

\begin{tabular}{llllll}
  \hline
  Year & Course Title                                & Grade \\
  \hline
  2013 & Mechanisms in Biomedical Sciences           & A     \\
  2013 & Current Topics in Genetics and Genomics     & A-    \\
  2014 & Genomics and Bioinformatics                 & A     \\
  2014 & Programming for Bioinformatics              & A     \\
  2014 & Advanced Cancer Biology                     & A     \\
  2014 & Molecular Medicine Survival Skills          & A     \\
  2014 & Genomics of Model Species and Humans        & A     \\
  2014 & Molecular Mechanisms of Signal Transduction & A     \\
  2015 & Research Ethics                             & Pass  \\
  \hline
\end{tabular}

% Skip a line between paragraphs.
\setlength{\parskip}{1.2ex}

% Set column sizes.
\newcommand{\firstcol}{3.5in}
\newcommand{\secondcol}{2.25in}
\newcommand{\thirdcol}{1.5in}
\newcommand{\allcols}{7.25in}


\end{document}
